\chapter*{ВВЕДЕНИЕ}
\addcontentsline{toc}{chapter}{ВВЕДЕНИЕ}

Распознавание видов физической активности человека является одним из актуальных направлений исследования в области машинного обучения, так как результаты распознавания необходимы при решении многих практических задач.

Студенты МГТУ им Н. Э. Баумана на первом курсе университета сталкиваются с дневниками самоподготовки на парах по физической культуре. Начиная с 2020 года эта система стала автоматизированный и студенты загружают видео со своими занятиями на сайт кафедры ФВ (физического воспитания), а преподаватели отсматривают эти видео. Если бы существовала система способная распознавать вид физических упражнений, выполняемых студентом, это бы ускорило процесс проверки дневников самоподготовки.

Также после некоторых операций пациентам необходимо во время реабилитации выполнять физические упражнения для более быстрого восстановления. В статье  \cite{med} производится выбор признаков, необходимых для классификации физической активности в телереабилитационной системе и рамках
этой системы разработан модуль классификации
видов физической активности на основе показаний датчиков
смартфона: акселерометра и гироскопа. Телереабилитация направлена на постоянный мониторинг физической активности человека.  

Цель работы -- разработать метод распознавания спортивных действий человека на видео с использованием локального дескриптора.

Для достижения поставленной цели необходимо решить следующие
задачи.
\begin{itemize}
	\item[---] Рассмотреть алгоритмы, с помощью которых можно реализовать метод распознавания спортивных действий человека.
	\item[---] Провести обзор существующих решений распознавания действий человека.
	\item[---] Разработать метод распознавания спортивных действий человека на видео с использованием локального дескриптора.
	\item[---] Спроектировать и реализовать ПО, демонстрирующее работу метода.
	\item[---] Исследовать метрики разработанного метода в зависимости от разметки данных.
	\item[---] Исследовать разработанный метод на применимость при работе с видео различной контрастности и качества.	
\end{itemize}
